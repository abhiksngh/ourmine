\begin{kasten}
    \section*{ \hspace{0.1cm} {\color{red} \underline{ISSUES WITH HYPERPIPES cont.}}}
    \large{

3. Outliers Ruin Scoring

\vspace{3 mm}
Imagine after 10,000 rows your min is 23 
and your max is 45. You encounter a new row where this attributes value is 
431. This causes your new max value to be 431. This can be a major pitfall 
if this is the only time a number this large is encountered for this class. 
 You have now expanded this HyperPipe to a max so large that this attribute 
loses its ability to classify.

    }
\end{kasten}

\begin{kasten}
    \section*{ \hspace{0.1cm} {\color{red} \underline{FIXING THE ISSUES}}}
    \large{

      1. Tied Classes
\vspace{3 mm}

Overwriting a class with another class of the same score puts a large emphasis on 
the order in which you score the classes. For this issue we 
decided to temporarily throw away the idea of classification 
and modified the code to return any classes who tied. This 
simple modification proved to us that this issue alone was a 
major factor in HyperPipes demise when put up against other 
classifiers. 

\vspace{3 mm}
      2. Over Fitting
\vspace{3 mm}

All of the algorithms below are triggered 
based on the percentage of times the class is returned but it is 
found not to be the actual class. Two modifiable attributes, detectAfter
and triggerWhen, indicate to the system how often to check for 
overfit and at what percentage we assume we have hit an overfit. 
Below we describe the different actions we take when an overfit
is detected:
\begin{enumerate}
\item Reset and Relearn: 
	When this overfit action is enabled it will keep track of the 
	last 100 rows encountered for each class. If an overfit is detected 
	the numerical bounds for that class are reset to nothing(start over).
	However, it then looks at the oldest 50 rows in the 100 row history 
	and relearns those into the pipe before continuing. Therefore when
	an overfit is detected we dont lose all classification ability on that
	pipe, we simply know less.
\item Revert Pipe on Detection:
	When the supporting code requests that the MultiPipe
	be checked for overfit if it is determined to not be overfit the 
	pipe is logged. If the pipe is determined to be overfit it will revert 
	the pipe to the previously logged MultiPipe.
	
\end{enumerate}

\vspace{3 mm}
      3. Outliers
\vspace{3 mm}

Below is a list of our outlier fixing 
strategies:
\begin{enumerate}
\item Weighted Means: What this fix 
does is it changes the scoring mechanism. In the pseudocode for 
Classify you will notice that it returns 1 if the current attribute 
value falls within the range of that given by the MultiPipe for that 
Attribute. The Weighted Means changes the 1 to a number between 0 
and 1 which represents the distance from the mean within the range. 
\item Outlier Detection Algorithm: 
Every time MultiPipes attempts to add experience to its numerical bounds
the algorithm calculates the z-score [2] for the value attempting to be added. 
If the Z-Score is greater than 1.96 then the value is not used to expand
the ranges within the hyperpipe.
\item Slowly Changing Ranges: 
The same fix we used in the outlier detection algorithm also allows for 
slowly changing ranges. Its important to realize that the value being
considered is added to the statistics for consideration in the z-score 
calculation and its addition to the statistics is not reversed if the
value is determined to be an outlier. This means that if the value 
determined to be an outlier occurs multiple times it is likely that 
it will eventually receive a z-score less than 1.96.
\end{enumerate}
    }
\end{kasten}

