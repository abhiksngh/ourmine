% THIS IS SIGPROC-SP.TEX - VERSION 3.1
% WORKS WITH V3.2SP OF ACM_PROC_ARTICLE-SP.CLS
% APRIL 2009
%
% It is an example file showing how to use the 'acm_proc_article-sp.cls' V3.2SP
% LaTeX2e document class file for Conference Proceedings submissions.
% ----------------------------------------------------------------------------------------------------------------
% This .tex file (and associated .cls V3.2SP) *DOES NOT* produce:
%       1) The Permission Statement
%       2) The Conference (location) Info information
%       3) The Copyright Line with ACM data
%       4) Page numbering
% ---------------------------------------------------------------------------------------------------------------
% It is an example which *does* use the .bib file (from which the .bbl file
% is produced).
% REMEMBER HOWEVER: After having produced the .bbl file,
% and prior to final submission,
% you need to 'insert'  your .bbl file into your source .tex file so as to provide
% ONE 'self-contained' source file.
%
% Questions regarding SIGS should be sent to
% Adrienne Griscti ---> griscti@acm.org
%
% Questions/suggestions regarding the guidelines, .tex and .cls files, etc. to
% Gerald Murray ---> murray@hq.acm.org
%
% For tracking purposes - this is V3.1SP - APRIL 2009 
\documentclass{acm_proc_article-sp}
\usepackage{algorithm}
%\usepackage{algorithmic}
\usepackage{algpseudocode}
%\usepackage{soul}

\usepackage{colortbl}

\newcommand{\boxplot}[5]{
\begin{picture}(100, 7)
\put(#1, 2){\line(0, 1){4}}
\put(#1, 4){\line(1, 0){#2}}
\put(#3, 4){\circle*{3}}
\put(#3, 4){\line(1, 0){#4}}
\put(#5, 2){\line(0, 1){4}}
\put(50, 3){\line(0, 1){4}}
\end{picture}
}

\usepackage{float}

\floatstyle{ruled}
\newfloat{program}{thp}{lop}
\floatname{program}{Program}

\begin{document}

\title{Multipipes: Exploring Disjuntive Classifications in Hyperpipes}

\numberofauthors{2} 
%
\author{
% The command \alignauthor (no curly braces needed) should
% precede each author name, affiliation/snail-mail address and
% e-mail address. Additionally, tag each line of
% affiliation/address with \affaddr, and tag the
% e-mail address with \email.
%
% 1st. author
\alignauthor
Aaron Riesbeck\\
       \affaddr{West Virginia University}\\
       \affaddr{100 Address Lane}\\
       \affaddr{Morgantown, WV 26505}\\
       \email{ariesbeck@theriac.org}
% 2nd. author
\alignauthor
Adam Brady\\
       \affaddr{West Virginia University}\\
       \affaddr{100 Fake St.}\\
       \affaddr{Morgantown, WV 26505}\\
       \email{adam.m.brady@gmail.com}
}
\date{12 October 2009}

\maketitle
\begin{abstract}

This paper explores classifiction with disjunctive sets using a modified form of HyperPipes~\cite{Eisenstein04} called MultiPipes. Rather than apply HyperPipes it's intended sparse datasets, we find that it's application to non-sparse, many-class datasets typically results in several tied classification scores which we then union into a disjunction. This union presents interesting possibilities in it's high accuracy in containing the target class. Although we initially cannot predict single classes, we find that these disjunctions often eliminate large portions of possible classes. Essentialy we aren't certain what the class is, but we are very certain of what the class is not. The rest of the paper explores two alternative strategies with MultiPipes. The first involves methods of reducing the disjunctive sets to single classifications. The second considers growing the disjunctive sets to optimize the accuracy of containment vs. set size.

\end{abstract}

\keywords{HyperPipes, disjoint sets, \LaTeX, multiple classes, indecisive learners} % NOT required for Proceedings


\section{Introduction to HyperPipes}

Background on hyperpipes, description of algorithm, benefits, trade-offs, relevant applications (spare datasets)


%\begin{figure}
\begin{program}
%\footnotesize
\begin{algorithmic}
\STATE{$HyperPipes := array[]$}
\STATE{$LinesGuessed := 0$}
\STATE{$LinesGuessedCorrectly := 0$}
\FORALL{Line in DataFile}
\IF{$HyperPipes is not empty$}
\STATE{$LinesGuessed++$}
\STATE{$BestScore := 0$}
\STATE{$BestClass := []$}
\FORALL{HPipe in HyperPipes}
\STATE{$HPipeScore := 0$}
\FORALL{Attribute in Line}
\IF{$Attr <= HPipe.Attr.max\; \&\&\; Attr >= HPipe.Attr.min$}
\STATE{$HPipeScore++$}
\ENDIF
\ENDFOR
\IF{$HPipeScore >= BestScore$}
\IF{$HPipeScore := BestScore$}
\STATE{$BestClass := BestClass + HPipe.class$}
\ELSE
\STATE{$BestClass := array(HPipe.class)$}
\STATE{$BestScore := HPipeScore$}
\ENDIF
\ENDIF
\ENDFOR
\IF{$BestClass\; contains\; Line.class$}
\STATE{$LinesGuessedCorrectly++$}
\ENDIF
\ENDIF

\STATE{$HyperPipes := AddExperience(Line,HyperPipes)$}
\ENDFOR 
\end{algorithmic}
\caption{HyperPipes Pseudo Code.}\label{Pseudocode}
\end{program}
%\end{figure}

%\begin{figure}
\begin{program}
%\footnotesize
\begin{algorithmic}
\Procedure{Euclid}{$a,b$}
\IF{$HyperPipes\; is\; not\; empty$}
\STATE{$LinesGuessed++$}
\STATE{$BestScore := 0$}
\STATE{$BestClass := []$}
\FORALL{HPipe in HyperPipes}
\STATE{$HPipeScore := 0$}
\FORALL{Attribute in Line}
\IF{$Attr <= HPipe.Attr.max\; \&\&\; Attr >= HPipe.Attr.min$}
\STATE{$HPipeScore++$}
\ENDIF
\ENDFOR
\IF{$HPipeScore >= BestScore$}
\IF{$HPipeScore := BestScore$}
\STATE{$BestClass := BestClass + HPipe.class$}
\ELSE
\STATE{$BestClass := array(HPipe.class)$}
\STATE{$BestScore := HPipeScore$}
\ENDIF
\ENDIF
\ENDFOR
\IF{$BestClass\; contains\; Line.class$}
\STATE{$LinesGuessedCorrectly++$}
\ENDIF
\ENDIF
\EndProcedure
\end{algorithmic}
\caption{HyperPipes Classify Pseudo Code.}\label{PseudocodeClass}
\end{program}
%\end{figure}


\begin{figure}[!t]
\renewcommand{\baselinestretch}{0.5}
\noindent
{\scriptsize
\begin{tabular}{c r  @{} c }
\multicolumn{3}{c}{Accuracy} \\\hline

Rank & Change  & 50\% \\
\hline
\rowcolor[rgb]{0.8,0.8,0.8}1 & flight2reuse, & \boxplot{17.700000}{3.500000}{21.200000}{12.100000}{33.300000} \\
\rowcolor[rgb]{0.8,0.8,0.8}2 & improveteam, & \boxplot{20.800000}{4.500000}{25.300000}{9.100000}{34.400000} \\
\rowcolor[rgb]{0.8,0.8,0.8}2 & none, & \boxplot{20.100000}{6.200000}{26.300000}{11.800000}{38.100000} \\
3 & reducefunc, & \boxplot{18.700000}{8.500000}{27.200000}{11.000000}{38.200000} \\
3 & improveprecflex, & \boxplot{22.400000}{7.700000}{30.100000}{11.800000}{41.900000} \\
3 & flight4reuse, & \boxplot{23.000000}{7.200000}{30.200000}{12.400000}{42.600000} \\
4 & relaxschedule, & \boxplot{21.000000}{7.500000}{28.500000}{18.400000}{46.900000} \\
4 & archriskresl, & \boxplot{21.900000}{7.900000}{29.800000}{10.200000}{40.000000} \\
5 & improvepmat, & \boxplot{21.200000}{10.000000}{31.200000}{15.100000}{46.300000} \\
6 & flight3reuse, & \boxplot{23.800000}{10.600000}{34.400000}{10.400000}{44.800000} \\
7 & reducequality, & \boxplot{30.600000}{5.500000}{36.100000}{11.400000}{47.500000} \\
8 & improvepcap, & \boxplot{29.600000}{9.800000}{39.400000}{8.100000}{47.500000} \\
9 & improvetooltechplat, & \boxplot{40.800000}{13.000000}{53.800000}{18.400000}{72.200000} \\
10& flight1reuse, & \boxplot{31.400000}{35.500000}{66.900000}{14.700000}{81.600000} \\

\end{tabular}
}
\caption{Accuracy of learning algorithms ignoring size of returned set}
\label{fig:accuracy}
\end{figure}


\begin{figure}[!t]
\renewcommand{\baselinestretch}{0.5}
\noindent
{\scriptsize
\begin{tabular}{c r  @{} c }
\multicolumn{3}{c}{Performance Score} \\\hline

Rank & Change  & 50\% \\
\hline
\rowcolor[rgb]{0.8,0.8,0.8}1 & flight2reuse, & \boxplot{17.700000}{3.500000}{21.200000}{12.100000}{33.300000} \\
\rowcolor[rgb]{0.8,0.8,0.8}2 & improveteam, & \boxplot{20.800000}{4.500000}{25.300000}{9.100000}{34.400000} \\
\rowcolor[rgb]{0.8,0.8,0.8}2 & none, & \boxplot{20.100000}{6.200000}{26.300000}{11.800000}{38.100000} \\
3 & reducefunc, & \boxplot{18.700000}{8.500000}{27.200000}{11.000000}{38.200000} \\
3 & improveprecflex, & \boxplot{22.400000}{7.700000}{30.100000}{11.800000}{41.900000} \\
3 & flight4reuse, & \boxplot{23.000000}{7.200000}{30.200000}{12.400000}{42.600000} \\
4 & relaxschedule, & \boxplot{21.000000}{7.500000}{28.500000}{18.400000}{46.900000} \\
4 & archriskresl, & \boxplot{21.900000}{7.900000}{29.800000}{10.200000}{40.000000} \\
5 & improvepmat, & \boxplot{21.200000}{10.000000}{31.200000}{15.100000}{46.300000} \\
6 & flight3reuse, & \boxplot{23.800000}{10.600000}{34.400000}{10.400000}{44.800000} \\
7 & reducequality, & \boxplot{30.600000}{5.500000}{36.100000}{11.400000}{47.500000} \\
8 & improvepcap, & \boxplot{29.600000}{9.800000}{39.400000}{8.100000}{47.500000} \\
9 & improvetooltechplat, & \boxplot{40.800000}{13.000000}{53.800000}{18.400000}{72.200000} \\
10& flight1reuse, & \boxplot{31.400000}{35.500000}{66.900000}{14.700000}{81.600000} \\

\end{tabular}
}
\caption{Peformance score of algorithms}
\label{fig:performance}
\end{figure}

\subsection{The Problem with HyperPipes}

On non-sparse datasets you get lots of ties, bra'h.


\section{HyperPipes 2.0: MultiPipes}

\subsection{Introducing MultiPipes}
We saw incredible potential in HyperPipes if we could reign in
the issues described above. One of the major advantages of HyperPipes
is its ability to classify a single row very fast. The Pipes 
condition can be stored and retrieved at a later time for 
classification unlike some other learners whose classification 
relies heavily on the actual contents of previous rows. Below we 
describe how HyperPipes was transformed into MultiPipes.

\subsection{Fixing the issues}
\subsubsection{Tied Classes}
When it was discovered that HyperPipes tended to choose classes 
towards the end of the classes list we investigated further to 
find that many classes were tieing against other classes. We 
then decided that this was unacceptable. Overwriting a class 
with another class of the same score puts a large emphasis on 
the order in which you score the classes. For this issue we 
decided to temporarily throw away the idea of classification 
and modified the code to return any classes who tied. This 
simple modification proved to us that this issue alone was a 
major factor in HyperPipes demise when put up against other 
classifiers. To recap, HyperPipes states that if 
the current score is equal to or greater than the best score 
set the best class to the current class. MultiPipes states that if the score is greater than the 
best score set the best class to an array containing only the 
current class. If the current score is equal to the best score 
then append the current class to the list of best classes.


\subsubsection{Over Fitting}
As described previously Hyperpipes has an over fitting issue 
when it learns too much. All of the algorithms below are triggered 
based on the percentage of times the class is returned but it is 
found not to be the actual class. Two modifiable attributes, detectAfter
and triggerWhen, indicate to the system how often to check for 
overfit and at what percentage we assume we have hit an overfit. 
Below we describe the different actions we take when an overfit
is detected:
\begin{enumerate}
\item Reset and Relearn: 
	When this overfit action is enabled it will keep track of the 
	last 100 rows encountered for each class. If an overfit is detected 
	the numerical bounds for that class are reset to nothing(start over).
	However, it then looks at the oldest 50 rows in the 100 row history 
	and relearns those into the pipe before continuing. Therefore when
	an overfit is detected we dont lose all classification ability on that
	pipe, we simply know less.
\item Revert Pipe on Detection:
	When this overfit action is enabled it will keep track of the previous
	ten MultiPipes. When the supporting code requests that the MultiPipe
	be checked for overfit if it is determined to not be overfit the 
	pipe is logged. If the pipe is determined to be overfit it will revert 
	the pipe to the previously logged MultiPipe
	
\end{enumerate}


\subsubsection{Outliers}
We have implemented some fixes to the problem of outliers. 
However, there are many other outlier fixing strategies that have 
not been implemented. Below is a list of our outlier fixing 
strategies and thier implemtation status:
\begin{enumerate}
\item Weighted Means: What this fix 
does is it changes the scoring mechanism. In the pseudocode for 
Classify (Program \ref{pro:Classify}) you will notice that it returns 1 if the current attribute 
value falls within the range of that given by the MultiPipe for that 
Attribute. The Weighted Means changes the 1 to a number between 0 
and 1 which represents the distance from the mean within the range. 
Two versions of this are explained further in Section 5.
\item Outlier Detection Algorithm: 
Every time MultiPipes attempts to add experience to its numerical bounds
the algorithm calculates the z-score\cite{Larsen01} for the value attempting to be added. 
If the Z-Score is greater than 1.96 then the value is not used to expand
the ranges within the hyperpipe.
\item Slowly Changing Ranges: 
The same fix we used in the outlier detection algorithm also allows for 
slowly changing ranges. Its important to realize that the value being
considered is added to the statistics for consideration in the z-score 
calculation and its addition to the statistics is not reversed if the
value is determined to be an outlier. This means that if the value 
determined to be an outlier occurs multiple times it is likely that 
it will eventually receive a z-score less than 1.96 and no longer be
considered an outlier and will subsequently be added to the min and max
for the hyperpipe.
\end{enumerate}



\section{Narrowing vs. Classifying}

\subsection{Introduction}
It may seem unfair to consider an algorithm that in many ways, cheats. While most classifiers are burdened with the task of predicting for a single class, it is possible for our algorithm to return all possible classes and claim victory in a meaningless fashion. While we will demonstrate that this extreme overclassifying is rarely the case, it stands to reason that comparisions cannot be made between the accuracy of our disjunctions and those of single classifiers.

However, just as HyperPipes had it's niche in sparse data, MultiPipes has a niche in reducing the space of possible classifications. Consider the task of matching a photo of a criminal to a database of millions. While returning a single match with prosecution strength accuracy would be ideal, reducing the matches to a handful of twenty or thirty would be nearly as useful. At that point the problem becomes tractable for a human, and our propensity for pattern matching can now supplement the learner. In other words, our best interest may not always lie in trusting the smartest learner in the room; the hubris of certainty ignores our own innate abilities of classification.

\subsection{Disjunctive Naive-Bayes}
Due to the success of turning HyperPipes into MultiPipes we thought we might be able to apply some of the same methods to Naive Bayes, specifically our Alpha measure described above. The idea being that we return any classes within a certain percentage of the highest scoring class. This method proved to be minimaly effective as the spread between the highest rated class and the next highest rated class tended to be very large meaning even a high alpha rarely caused the returned set to be larger than a single classification.
\subsection{Measuring Performance}


\section{Preliminary Results}

\subsection{Disjoint Learning}

nb vs. multipipes on >1 dataset (incremental)
nb vs. multipies on >1 dataset (batch)
(see menzies.us/iccle/?nb chart for dataset scores comparison)

size of sets returned relative to number of total classes



\subsection{Breaking the Ties}

While we believe the mutliple class result of HyperPipes is 
an interesting discovery we understand the need for a complete 
classification system. We have come up with two methods for
attempting to classify the result of MultiPipes into a single 
class and have come up with two different approaches. One 
approach was discovered when attempting to relieve outliers.
\subsubsection{Weighted Distance from Mean}
There are two methods we use for calculating the distance from 
the mean in our algorithm. As stated above in "Fixing Outliers" 
we noted that an attribute range scores a 1 if it falls within 
the min and max for that attribute in a class. Our modification, 
as stated above was to use a normalized calculation to determine 
a value between 0 and 1 based on the attribute values distance 
from the mean attribute value in that class. Two calculations 
were discussed and they are as follows:
\begin{enumerate}
\item Weighted Distance From Mean 1: In this calculation it was 
decided that the distance from mean should be calculated as:
\begin{equation}
  WghtDist=\frac{(Max-Min)-|(Mean-Val)|}{(Max-Min)}
\end{equation}
where Max, Min, and Mean are the max, min, and mean values for 
the attribute in the HyperPipe for the class currenlty being 
scored and Val is the current attribute value in the row being 
classified. 
\item Weighted Distance From Mean 2: The calculation in this 
method is slightly different from the one above:
\begin{equation}
  BigGap = max((Max-Mean),(Mean-Min))
\end{equation}
\begin{equation}
  WghtDist=\frac{(BigGap)-|(Mean-Val)|}{(BigGap)}
\end{equation}
where Max, Min, and Mean are the max, min, and mean values for 
the attribute in the HyperPipe for the class currenlty being 
scored and Val is the current attribute value in the row being 
classified.
\end{enumerate}



graph of weighted distance classification accuracy


\subsubsection{Centroid Assumption}
One method we implemented for classifying to a single class was
a centroid assumption function we created. The idea behind this
algorithm is that when a tie is formed we calculate the class 
which is believed to be at the center of the returned classes. 
This center class is determined by scoring each HyperPipe's 
attribute overlap with other HyperPipes. For each HyperPipe we 
sum the number of attributes in that HyperPipe whose min and max 
overlap with another classes min and max. In other words, for a 
single attribute in a HyperPipe the maximum score for that 
attribute is the number of HyperPipes minus 1. The maximum score 
for the HyperPipe as a whole is the number of attributes times 
the number of hyperpipes minus 1. After calculating this sum for 
each HyperPipe the class within HyperPipe with the highest score 
is chosen as the classification. Pseudocode:
\begin{program}
\begin{algorithmic}
\Procedure {FindCentroid}{$HyperPipes$$}
\State{$BestScore := 0$}
\ForAll{MainPipe in HyperPipes}
\State{$CurrentPipeScore := 0$}
\ForAll(TestPipe in HyperPipes)
\If{$MainPipe != TestPipe$}
\ForAll(Attr in MainPipe)
\If{$((MainPipe.Attr.min > TestPipe.Attr.min) && (MainPipe.Attr.min < TestPipe.Attr.min)) || ((MainPipe.Attr.max < TestPipe.Attr.max) && (MainPipe.Attr.max > TestPipe.Attr.max))$}
\State {$CurrentPipeScore++$}
\EndIf
\EndFor
\EndIf
\EndFor
\EndFor
\EndProcedure
\end{algorithmic}
\caption{HyperPipes Classify Pseudo Code.}\label{PseudocodeClass}
\end{program}
graph of centroid learning results



\subsection{Casting a wider net}

During testing it was determined that MultiPipes could easily be used 
as a reliable ???narrower??. We found that on some datasets we were 
able to acheive a very high reliability. That is, our resulting output 
set contained the actual class more than 90% of the time. However, we
found that on some other datasets the reliability was quite less. 
When searching for a solution we realized we needed a way to expand 
the returned dataset. This would increase the chance that the desired 
class is included in the returned set. The reason to expand the 
returned set is so we can improve the reliability to say with greater
confidence that the returned set is complete and accurate. Our next 
step is to train the learner as to which alpha value it should use. 
We would specify a goal reliability and expand the alpha slowly until 
that reliability is reached. If you look at Program 2 above you will 
notice that we add anything to the BestClass list that has a score 
equal to the best score found. Each score is a fraction and the best 
possible score is 1. When you apply an Alpha of .1 it basically 
states that if the the current score is greater than the current best 
score minus .1 append it to the list. This requires one run through the 
list of HyperPipes to find the best score. It then calculates the score 
to beat as \begin{math}ScoreToBeat = BestScore - Alpha\end{math}. On its second run through the 
HyperPipes any HyperPipe with a score greater than or equal to the 
\begin{math}ScoreToBeat\end{math} is included in the resulting set. 

Results of expanding alpha (graph)

Analysis of growth in enclosure with alpha changes



\section{Conclusions}

Lamentably, on many-class (approx. 20) datasets Multipipes does not perform significantly better than Naive Bayes when scored using our performance measure to penalize overclassification. Even worse, when tested on datasets with between two and twenty-four classes Multipipes does significantly worse.

%\vspace{3

Also, Multipipes tends to over-classify despite our efforts to reign in overfitting. On average Multipipes returns two-thirds of the possible classes, the average number of classes for a dataset being seven. While there are potential benefits from reducing the space of possible classes, the accuracy of our disjunctions is only slightly better than Naive Bayes'. Increasing the "reach" of our disjunctions using alpha values improves this accuracy to over 99\%, but at that point Multipipes is typically returning every single possible class.



% The following two commands are all you need in the
% initial runs of your .tex file to
% produce the bibliography for the citations in your paper.
\bibliographystyle{abbrv}
\bibliography{multipipes}  %the name of the Bibliography in this case
% You must have a proper ".bib" file
%  and remember to run:
% latex bibtex latex latex
% to resolve all references

\balancecolumns
\end{document}
